\section{金融市场基础设施原则与分布式账本}\label{sec:pfmi}

2008年金融危机爆发以来,国际社会对于构建安全、高效的金融基础设施提出了更高的要求。十多年来诞生了两方面的成果。

一个是2012年4月国际清算银行(BIS)发布的《金融市场基础设施原则》(PFMI)\cite{pfmi}。
针对金融市场基础设施的建设,PFMI分析了金融系统的主要风险(系统性风险、法律风险、流动性风险、业务风险、托管与投资风险、运营风险),
从9个方面(总体架构、信用与流动性风险管理、结算、证券存管与交易、违约管理、运营风险管理、准入机制、效率、透明度)提出了24条指导原则。
PFMI从金融系统顶层设计的角度,进一步确立了全球金融市场基础设施建设标准,为世界各国开展相关工作指明了方向。

另一个是以比特币为代表的分布式账本技术。
在2008年,中本聪(Satoshi Nakamoto)发表了比特币的白皮书,描述了一种不依赖于单一信用机构的记账技术。
这种后来被称为分布式账本的新技术迅速发展,各种数字货币与公有链竞相出现,构建了一个全新的金融服务生态,引起国际社会特别是金融界的广泛关注和高度重视。
分布式账本提供了公共的账本管理平台,互相独立的金融机构可以在同一个账本上彼此协作。
这种公共账本模式为对金融系统的深远影响体现在以下几个方面:

\begin{enumerate}
    \item 效率:跨行、跨境交易可以在公共账本上一次性完成结算,缩短了交易环节,简化了交易流程,降低了交易摩擦。
    \item 风险:由于缩短了中间转账的环节,降低了对于中间人的信用风险、流动性等风险。
    \item 透明度:资金流数据集中记录在一个公开、透明的账本上,提高了交易数据的完整性和可访问性。相对于监管多个分散的账本,监管一个公共账本会大幅降低监管的难度。无论是对于监管层、还是金融机构来讲,监管合规的成本也会大幅降低。
    \item 自动化:利用智能合约,将金融资产映射为可编程的Token,提高资产发行、销售、流转、托管等业务的电子化、自动化
    \item 标准化:众多金融机构使用同一个账本,便于统一彼此之间的通信协议、数据格式,提高金融行业规范化、标准化程度
    \item 全球化:分布式账本建立在全球化的互联网基础上,而不是封闭的专用金融网络。各个国家和地区可以共享同一套系统,便于搭建国际化的金融系统。
    \item 避免重复建设:采用国际化的公共账本,对于金融机构,可以减轻维护独立账本的负担;对于国家和地区,缓解了各自建设金融专用网络的需求。
\end{enumerate}

总之,对重塑金融业未来发展,PFMI和分布式账本都有重大的价值。
PFMI 是自顶向下、为金融系统的顶层设计提供了指导原则;分布式账本是自底向上、在全新的账本管理技术的基础上,重构整个金融系统。
PFMI与分布式账本技术的出现有相同的时代背景,都是源自于次贷危机的影响,它们的目标都是要建立更加安全、高效、可靠、可信的金融系统。
这二者虽然方法和方式不同,但是同源同宗,本质上是互补的。

