\section{账本三角不可能原理}\label{sec:triangle}
根据PFMI的指导原则,金融基础设施应该接受中央银行、市场监管者或者其它管理部门适当、有效的管理、监管和监督。
为了保证金融系统的安全和稳定,各国央行和监管机构对金融机构提出了Know-Your-Customer(KYC)、Anti-Money-Laundering(AML)、Counter-Financing-of-Terrorism (CFT)等合规要求。

但是在实际应用过程中,分布式账本技术在这方面一直比较弱,甚至有些产品的设计目标就是抗审查、抗监管。
从一定程度上助长了洗钱等金融犯罪行为。
根据日本警方报告\cite{jp_report},2018年加密货币洗钱案件增加了十倍。
2017年,日本国家警察局发现了不到700起加密洗钱事件;2018年,他们发现了7,000多起同类案件。

%分布式账本要成为金融基础设施,在保留其技术优势的同时,还要满足金融监管的要求。

这个矛盾的根源在于分布式账本作为共享账本,满足了账本数据的公开性,无法再满足完备性。
一般来讲,在金融系统中,无论是中心化账本还是去中心化账本,都无法同时满足公开性和完整性。
完整性、公开性和隐私性的定义如下:

\begin{definition}[去中心化]
    账本数据保存在多个不同的节点中,便于其它节点验证账本数据,并且保证账本的可信度。
\end{definition}

\begin{definition}[完整性]
    按照PFMI的需求:金融机构或者金融监管方需要完整的账户信息和交易信息、以及其它用于验证交易是否合规的相关信息。
\end{definition}

\begin{definition}[隐私性]
    只有交易的参与方与金融监管方才能了解交易的相关隐私数据,其它人员无法根据公开数据推导出交易隐私信息。
\end{definition}

\begin{itemize}
    \item[\dag] \textbf{去中心化}:
    账本数据保存在多个不同的节点中,便于其它节点验证账本数据,并且保证账本的可信度。

    \item[\dag] \textbf{完整性}:
    按照PFMI的需求:金融机构或者金融监管方需要完整的账户信息和交易信息、以及其它用于验证交易是否合规的相关信息。
    
    \item[\dag] \textbf{隐私性}:
    只有交易的参与方与金融监管方才能了解交易的相关隐私数据,其它人员无法根据公开数据推导出交易隐私信息。

\end{itemize}

对于中心化账本来讲,账本数据和交易数据是不公开的,存在信息不透明、有暗箱操作的空间,所以依赖于账本管理者的信用背书。但是由于账本数据不公开,可以保存用户和交易的隐私,有利于保存完整的交易信息,便于验证交易是否合规。

对于去中心化账本来讲,账本数据对于所有节点是公开的,其它节点可以验证所有交易数据,不需要可信的中介,依然可以保证账本数据的可信性。但是由于公开账本无法保护用户的隐私,只能将不敏感的数据上链,其它敏感数据(用户身份、交易背景)无法在链上保存。导致链上交易数据与链下背景信息的脱节,金融监管与合规非常困难。

总之,账本数据的不完整性是分布式账本本身固有的特征。如果将分布式账本应用于金融基础设施,满足 PFMI 以及各国金融法律对于监管合规相关要求,必须要创造性的解决这个难题。


本质上这要求金融机构了解账户的真实身份信息与真实的贸易背景数据,便于监控并且打击不法分子的金融犯罪行为。
目前的主流分布式账本平台普遍无法满足合规、监管需求,具体的来讲,表现在以下几个方面:

\begin{itemize}
    \item 用户真实身份信息的缺失
    \item 交易背景数据的缺失
    \item 交易过程中缺少合规验证环节
\end{itemize}
