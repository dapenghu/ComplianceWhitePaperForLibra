\section{账户数据结构}\label{sec:account_structure}

根据第5节的讨论,由于审核权与记账权的分离,账户数据也分为两部分:
脱敏的资产数据存储在分布式账本上,并且公开,任何人和节点都可以查看,这部分数据包括账户ID,账户资产等;
另一部分数据由合规机构中心化管理,不对外开放,只有交易的相关方或者监管机构才有权查看,主要包括用户的真实身份相关的隐私数据。
两部分数据虽然是分离存储和管理的,但是在逻辑上是不可分割的,彼此保持完整性和一致性。本节讨论如何将这两部分数据集成在一起。

\subsection{链下账本数据}
链下的个人、企业身份信息包含多个维度的数据,不仅包括文本类数据:姓名、性别、出生日期、国家、城市、住址、身份证号码、护照号码、电话号码、邮箱,还包括多媒体数据:照片、音频、视频等等。
在完成 KYC 审核的过程,用户将这些数据提交给合规机构。
根据每个司法辖区以及金融业务的具体情况,各个合规机构可以自行决定用户必须提交哪些数据,以及这些数据的存储格式。

在 AML 等合规检查过程中,合规机构不仅需要交易发送方的,同时还需要接受方的身份信息。
一般情况下交易双方注册的合规机构不是同一家,那么这两个合规机构需要互相通信、交换双方的身份信息。
这些字段应当使用 ISO 标准的编码方式,以便于数据格式和通信协议的统一。
下表列出一些相关的ISO标准。更加具体的规范超出了本白皮书的范围,Libra 会另有文档详细描述相关的规范。

\begin{table}[h!]
    \caption{数据格式标准} 
    \label{tab:data_format}
    \small % text size of table content
    \centering % center the table
    \begin{tabular}{lrl} % alignment of each column data
        \toprule[\heavyrulewidth]\toprule[\heavyrulewidth]
        \textbf{字段} & \textbf{编码长度(字符)} & \textbf{编码格式} \\ 
        %\midrule
        \hline
        语言 & 2 & ISO 639-1 \\
        \hline
        国家 & 3 & ISO 3166-1 alpha-3 \\
        \hline
        日期 & 10 & ISO 8601 date-only format \\
        \hline
        电话 & 可变长度 & E.164 \\
        \hline
        职业 & 3 & ISO ISCO08 \\
        \bottomrule[\heavyrulewidth] 
    \end{tabular}
\end{table}

\subsection{链上账户数据}
类似于以太坊,Libra 账本也是基于账户的数据模型,账户在 Libra 中是最核心的数据结构。
一个账户地址是一个256位二进制整数,整个账本是以账户地址为索引的 key-value 类型数据库。
地址对应的 value 是一个数据集合,记录账户的状态和相关数据。
一般来讲包括 Resource 和 Module 两类数据,Module 是智能合约语言 Move 的二进制代码,Resource 是资产类数据。
对于本文来讲,我们更加关心合规相关的数据,具体来讲一个账户还包含以下属性:

\begin{itemize}
    \item[\dag] \textbf{IdentityRoot}: 真实身份信息的 Merkle Root

    通过KYC审核之后,合规机构将用户真实身份的所有字段组织到一个 Merkle Tree 中,获得对应的 Merkle Root,存储在这个属性里。这种数据结构有两个优点:
    \begin{enumerate}
        \item 可以将这个 Merkle Root 看做用户身份数据的指纹,用于证明合规机构没有篡改用户身份信息。
        \item 在有些场景中,并不需要披露所有的身份信息,可以有选择地公开某一部分内容,通过 Merkle Tree 的路径证明验证数据的一致性。
    \end{enumerate}

    \item[\dag] \textbf{Validator}: 合规机构账户

    这个属性是一个合规机构的 256 位账户地址。
    除了 Libra 协会的系统账户之外,任何一个账户都是经过KYC注册之后才能建立。这个属性是对应的合规机构的账户地址。
    对于合规机构来讲,这个属性是 libra 协会的系统账户地址。

    \item[\dag] \textbf{Domain}: 域名

    这个属性是字符串格式的域名。它是可选的,对于普通用户来讲,其账户信息是保密的,所以此属性的值为 NULL。
    对于具有公信力的权威机构、或者接受公众监督的金融服务机构,他们的账户要提供真实的域名。
    如果一个账户设置了域名,公开真实身份,这个账户被称为\textbf{知名账户}。例如:

    \begin{itemize}
        \item Libra 协会成员必须知名账户,负责 Libra 社区治理、验证节点的管理和维护。
        \item 合规机构必须是知名账户,与其它合规机构合作共同完成 AML 等合规验证。
        \item 资产发行方必须是知名账户,披露资产的真实信息。
        \item 慈善机构可以是知名账户,它管理的善款接受公众监督,规避内部运营风险。
    \end{itemize}
\end{itemize}

\subsection{知名账户的配置文件 libra.toml}
知名账户明确通过域名标识自己的真实身份。
除了域名之外,其它的相关信息记录在一个名为 $libra.toml$ 的配置文件中。
$libra.toml$ 文件是用 TOML 格式编写的,这是一种简单且广泛使用的配置文件格式。
$libra.toml$ 文件对应的URL地址是\url{https://<Domain>/.well-known/libra.toml}。
比如说一个知名账户的域名是 example.com,那么对应的配置文件 URL 是:\url{https://example.com/.well-known/libra.toml}。
注意:这里使用的是 HTTPS 协议,对应的Web服务器上必须部署 libra 协会为此机构颁发的数字证书。

配置文件 $libra.toml$ 包含非常丰富的内容,不但包括完整的机构相关说明:机构的名称、办公地址、法人代表、公司邮箱、Twitter 等等;
还可以包括在链下提供的服务相关信息。比如

\begin{itemize}
    \item 如果账户代表一个 Libra 的记账节点,$libra.toml$ 包含节点的相关信息
        \begin{itemize}
            \item DISPLAY\_NAME:用于浏览器和其它API的可读显示名称。
            \item PUBLIC\_KEY: 与该验证节点关联的公钥。
            \item HOST: 连接验证节点的域名+端口号
        \end{itemize}
        例如:
        \begin{lstlisting}[caption={记账节点配置信息}, label={lst:validator}]
        [MINNER]
        DISPLAY_NAME="Visa USA"
        HOST="usa.visa.com:30307"
        PUBLIC_KEY="GAOO3LWBC4XF6VWRP5ESJ6IBHAISVJMSBTALHOQM2EZG7Q477UWA6L7U"
        \end{lstlisting}
        
    \item 如果账户代表一个合规机构,$libra.toml$ 可以声明合规服务相关的信息,比如
        \begin{itemize}
            \item DISPLAY\_NAME:用于浏览器和其它API的可读显示名称。

            \item SIGNING\_KEY:合规机构的公钥,使用此公钥验证合规机构的签名。
            
            \item KYC\_SERVICE:合规机构对外提供的 KYC 服务地址,客户通过此服务提交材料完成 KYC 审核

            \item AUDIT\_SERVICE:合规机构对外提供的合规服务地址,客户通过此服务提交交易数据,完成AML/CFT检查。

        \end{itemize}
        例如:
        \begin{lstlisting}[caption={验证机构配置信息}, label={lst:validator}]
        [VALIDATOR]
        DISPLAY_NAME="Citi Bank"
        KYC_SERVICE="https://www.citi.com/kyc"
        AUDIT_SERVICE="https://www.citi.com/aml"
        SIGNING_KEY="SVJMSBTALHOQM2EZG7Q477UWA6L7UGAOO3LWBC4XF6VWRP5ESJ6IBHAI"
        \end{lstlisting}

    \item 如果账户是一个数字资产智能合约,$libra.toml$ 可以描述此数字资产的相关信息
        \begin{itemize}
            \item ASSET\_CODE: 此数字资产的代码(3-5字母)
            
            \item ASSET\_DESCRIPTION:此数字资产的描述

            \item ASSET\_LOGO:数字资产的logo图片URL

            \item ASSET\_GITHUB:此数字资产智能合约的源码地址

            \item ASSET\_ISSUER: 此数字资产的发行方账户地址
        \end{itemize}
        例如:
        \begin{lstlisting}[caption={资产配置信息}, label={lst:validator}]
        [ASSET]
        ASSET_CODE="LBRT"
        ASSET_DESCRIPTION="Libra Token"
        ASSET_LOGO="https://api.cryptorank.io/static/img/coins/libra1561525522731.png"
        ASSET_GITHUB="https://github.com/libra"
        ASSET_ISSUER="Libra Association"
        \end{lstlisting}
\end{itemize}

$libra.toml$ 文件提供了一种账户信息扩展机制,可以把不适合上链的公开信息和公共服务API注册在此文件里面。
其它用户、Dapp或者金融机构可以通过解析此文件发现对应的信息。
