\begin{abstract}
Libra 的愿景之一是利用分布式账本技术构建全球化的金融基础设施,
必须要满足了解客户(KYC), 反洗钱(AML)、反恐怖主义融资(CFT)等金融监管与合规要求。
本文阐述了 Libra 关于金融监管与合规的相关架构设计。
首先我们指出:分布式账本技术必须与金融市场基础设施原则(PFMI)相结合,才能构建安全、高效、规范的金融系统;
其次,根据账本数据的去中心化、完整性、隐私性三角不可能原理,得出结论:分布式账本必须与中心化的身份管理、合规审核系统集成,才能满足监管与合规的需求。
再次,介绍了加拿大央行、欧洲央行、日本央行、新加坡金融管理局等相关工作的经验。
最后,从组织安排、账本数据结构、准入与账本体系、合规协议4个方面阐述了合规框架设计。

\end{abstract}
