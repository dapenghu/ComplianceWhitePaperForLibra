\section{组织安排}\label{sec:arrangement}
根据“挑战”一章的结论,单纯的去中心化账本管理系统满足了账本的公开性,牺牲了完整性;反之,单纯的中心化账本管理系统满足了完整性,牺牲了公开性。从这个结论出发,去中心化账本系统必须结合中心化账本系统,其中分布式账本系统负责管理记账权,保证账本的可信度与透明度;中心化账本负责管理用户真实身份、交易背景数据等隐私数据,补充账本数据的完整性。这种设计路线的要点是:满足监管合规的要求的前提下,最大限度的保留分布式账本的技术优势。

根据这个总体设计思路,下面从组织安排、账本数据结构、合规验证协议、智能合约、准入规则等多个角度详细阐述 Libra 合规架构设计。

在传统金融系统中,银行等金融机构接受用户的委托,以负债的形式托管用户的资产,同时拥有记账权、知情权和审核权。也就是说,银行不但负责代理用户处理贷记/借记事务,同时还要知晓用户的身份信息、交易背景信息,以及根据金融监管的要求审核每一笔交易的合规性。

在 Libra 生态中,金融机构依然扮演重要角色,差别在于用户的资产记录在 Libra 分布式账本上,记账权以去中心化的方式分配给各个验证节点,所以金融机构不再拥有记账权。但是依然拥有知情权和审核权,它负责保护用户的隐私,为 Libra 生态提供了合规服务。这种只负责合规、不负责记账的新金融机构称之为\textbf{合规机构}。它为 Libra 分布式账本提供预言机服务,连接链上的交易数据和链下的背景数据,负责账本数据的完整性。

合规机构作为一种新的金融基础设施,可以是原来的银行、保险公司、交易所等金融机构,也可以是电信运营商、电子商务平台等非传统金融机构,甚至监管机构可以直接作为合规机构。合规机构是中心化的,需要信用背书,所以它必须持有相关牌照的经营实体,接受当地辖区的金融监管,能够有效的保护用户的隐私、行使合规服务的相关责任。

从用户的角度来讲,资产的所有权和其它公链差异不大,数字资产记录在 Libra 分布式账本上,通过控制账户对应的私钥明确资产的所有权。所有资产的处置都必须要通过对交易做数字签名才能执行。不同于比特币、以太坊等其它分布式账本,Libra 的账户有注册机制。每一个用户都需要向当地辖区的一个合规机构注册自己的真实身份,审核通过之后,然后在合规机构的协助下建立 Libra 账户。只有合规机构知晓账户对应的真实身份信息,对于其它人是保密的。另外每一笔交易也都需要合规机构的参与,验证交易是否合规。

从监管层的角度来讲,合规机构负责保证 Libra 平台上资产和交易背景数据的完整性、真实性、有效性。是执行反洗钱、反恐怖融资的等相关金融政策的主体。通过对少数合规机构的监督和管理,能够实现去中心化金融系统上的中心化监管,降低监管成本,有效的推动相关金融政策的实施,维护金融稳定。

从 Libra 协会的角度来讲,合规机构是 Libra 构建金融基础设施的重要组成部分,是连接用户、金融监管系统的桥梁。 Libra 借助于合规机构拓展用户群体,提供支付、借贷、证券等合规金融服务;同时,Libra 通过合规机构解释并执行各个辖区的法律法规,监控、抵制在 Libra 系统上的金融犯罪。Libra 协会负责为合规机构提供技术支持与相关咨询服务。由于合规机构必须是有信用背书的法人实体,也必须是实名制的。在开展业务之前,合规机构需要向 Libra 协会注册。Libra 根据当地辖区的法律要求,审核合其资格与经营范围,并且建立 Libra 账户。由于合规机构的特殊性和重要性,Libra 协会为合规机构发放 X.509 数字证书,任何人可以通过此证书验证它的真实性,防止被钓鱼欺诈。





