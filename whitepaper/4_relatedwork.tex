\section{相关工作}\label{sec:related}

在PFMI指导下,各国央行与国际金融机构积极探索分布式账本在金融基础设施上的应用。2016至2017年,世界清算银行(BIS)\cite{bis_dlt}、美联储\cite{fr}、美国存管信托和结算公司(DTCC)\cite{dtcc2016}、欧洲央行\cite{euro}等机构相继发表了研究报告,从理论上分析了分布式账本改造金融市场基础设施的可行性。不仅如此,多家央行与区块链初创公司合作,针对支付、清算、结算等多个不同金融场景,创建了概念验证项目,根据实验结果调查分布式账本的应用前景。下面介绍几个相关成果。

加拿大银行、Payments Canada、TMX集团和埃森哲合作发起了 Jasper 项目,旨在研究如何通过分布式账本技术来优化加拿大的金融系统。
该项目于2016年启动,迄今已经历了三个实验阶段。
第一阶段和第二阶段\cite{jasper2}分别使用以太坊和R3 Corda 构建了银行间支付原型系统,以调查中央银行发出的数字收据的使用,而非现金支持结算付款。
这项工作的关键结论:
1. 与现存的 LVTS 系统相比,分布式账本技术为银行间支付带来的总体效益并不是很明确,但是可以优化流动性节约机制。
2. 当多个资产在同一个分布式分类账系统上结算时,才有可能实现显着的效率提升。
依据第二个结论,Jasper 的第三阶段\cite{jasper3}在 R3 Corda 的基础上调查了证券结算DVP系统。
测试的结论是:在满足隐私性、可扩展性等要求的前提下,分布式账本能够用于证券交易的关键功能:现金、股票的抵押、赎回,以及结算。

%2017年6月,加拿大央行出了一份报告\cite{jasper1},对R3 CEV的Corda平台进行批评,并用金融市场基础设施原则(Principles of Financial MarketInfrastructures, PFMI)原则评估分布式账本系统,其中提到一个概念就是“透明度不够”(Lack of transparency)。

新加坡金融管理局(MAS)于2016年11月16日宣布,它与R3、多家金融机构合作开展概念验证项目 Ubin。目前该项目有4个阶段。
第一阶段\cite{mas1}和第二阶段\cite{mas2}调查分布式账本在银行间支付系统的作用。
基本结论是:分布式账本( 特别是Corda,Hyperledger Fabric 和 Quorum) 能够应用于银行间的实时总结算(RTGS),而且满足容量,流动性节约机制,僵局解决方案,安全性,不变性和弹性等关键指标。
第三阶段\cite{mas3}和新加坡证券交易所合作,研究分布式账本在证券结算DVP系统的应用。
研究表明:分布式账本在规则集成(Rulebook integration),压缩结算周期(Compression of settlement cycle),
仲裁设计(Arbitration design),增强安全性和隐私性(Enhanced security and privacy),
流动性和市场结构(Liquidity and market structure)方面有很大发展潜力。
第四阶段\cite{mas4}与加拿大银行,英格兰银行合作,评估银行间跨境支付和结算的替代模式。

%Jasper 的研究报告指出,分布式账本技术的应用有助于扩展加拿大金融创新,并有可能在某一天帮助促进国内和国际金融市场一体化。
%但是同时指出如果要有效地将分布式账本纳入到金融系统,在很多方面必须符合PFMI的要求。比如说透明度和隐私。
%Jasper 发现 R3 Corda 透明度不够,即不能很快找到账户信息而需要进行搜寻。
%透明度应该是分布式账本的长项,来源于公共账本特征,每一个参与节点都存有同样的数据。
%但Corda并没有共享账本,每个节点可能存不同信息,加拿大央行认为这不是好的设计。

2016年12月, 欧洲央行、日本央行联合发起了联合研究项目恒星(Stella), 致力于研究分布式账本技术对FMI的机遇与挑战。
它使用 Corda,Elements 和 Hyperledger Fabric 开发了多个原型。
同时用 PFMI 原则来评估这些分布式账本系统。
此项目已经完成了3个阶段,分别研究分布式账本在大额支付系统\cite{stellar1}、证券交易结算系统\cite{stellar2}、同步跨境支付系统\cite{stellar3}的前景。
这份研究指出:从技术上讲,分布式账本有潜力改善现有金融系统,但是依然缺少成熟度,需要在法律、合规等方面进一步的评估和加强。

%南非储备银行的 Khokha 项目\cite{sarb} 调查了基于 Quoram 的银行间支付系统。
%调查结果显示,在地理位置分散的云环境下,系统的延时能够控制在2秒之内。

这些研究表明分布式账本技术可以应用于多种金融基础实施,但是在不同的领域、不同的国家,收益有差异。
需要决策者根据本国国情谨慎评估、并且从整体上统筹规划。
从技术成熟度来看,现有分布式账本平台还有很多问题,离PFMI的要求还有差距。
比如在合规性、透明度方面,被评估的几种平台(Ethereum, Corda,Hyperledger Fabric 和 Quorum)普遍需要改进,它们不支持快速找到账户身份信息和交易背景信息、进行合规审查。
业界普遍关注分布式账本的可扩展问题,而忽视了合规问题。殊不知,合规问题更重要。
Libra 也是犯了类似的错误,系统的可扩展性很好,但是在合规方面做得不够,受到了参众两院和金融监管层的严厉批评。

%结构性的问题不是一两天就能解决,也不是一两个月就能够解决,结构性的问题可能一两年都无法解决,而且解决方案有可能是重新做一个系统,因为原来系统结构不能匹配PFMI原则。
