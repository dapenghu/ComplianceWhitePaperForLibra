\section{相关工作}\label{sec:related}

在PFMI指导下,各国央行与国际金融机构积极探索分布式账本在金融基础设施上的应用。2017年,国际清算银行(BIS)发表了一份报告 \cite{bis_dlt},阐述了分布式账本在PFMI下需要满足的需求与功能。不仅如此,在过去的几年中,多国央行组织了多项实验分布式账本研究项目,针对支付清结算的各个不同应用场景,评估分布式账本的应用前景,遗憾的是当前多个主流的分布式账本平台并未充分满足PFMI的要求。

\textbf{Jasper项目 - 加拿大央行}
    
Jasper项目是 Payments Canada,加拿大央行,TMX集团和埃森哲之间的合作项目,于2015年启动,旨在了解 分布式账本技术(DLT) 如何通过开发基于分布式账本技术来改变加拿大的金融系统。该项目迄今已经历了三个实验阶段。Jasper项目的第一阶段使用以太坊平台构建分布式账本原型和概念验证系统,以调查中央银行发出的数字收据的使用,而非现金支持结算付款。第二阶段使用R3的开源分布式账本平台Corda构建了一个原型,以进一步探索:分布式账本如何改变中心化系统的结构和运行方式,分布式账本系统是否符合国际标准,以及对支付系统政策的任何潜在影响。第三阶段探讨了分布式账本对更广泛的加拿大金融市场基础设施的影响和潜在价值。


2017年6月,加拿大央行出了一份报告\cite{jasper1},对R3 CEV的Corda平台进行批评,并用金融市场基础设施原则(Principles of Financial MarketInfrastructures, PFMI)原则评估分布式账本系统,其中提到一个概念就是“透明度不够”(Lack of transparency)。

Jasper 的研究报告指出,分布式账本技术的应用有助于扩展加拿大金融创新,并有可能在某一天帮助促进国内和国际金融市场一体化。但是同时指出如果要有效地将分布式账本纳入到金融系统,在很多方面必须符合PFMI的要求。比如说透明度和隐私。Jasper 发现 R3 Corda 透明度不够,即不能很快找到账户信息而需要进行搜寻。透明度应该是分布式账本的长项,来源于公共账本特征,每一个参与节点都存有同样的数据。但Corda并没有共享账本,每个节点可能存不同信息,加拿大央行认为这不是好的设计。

\textbf{Stellar项目 - 欧洲、日本央行}

2016年12月, 欧洲央行、日本央行联合发起了联合研究项目恒星(Stella), 致力于研究分布式账本技术对FMI的机遇与挑战。它使用 Corda,Elements 和 Hyperledger Fabric 开发了多个原型。同时用 PFMI 原则来评估这些分布式账本系统。此项目已经完成了3个阶段,分别研究分布式账本在大额支付系统、证券交易结算系统、同步跨境支付系统的前景。这份研究指出:从技术上讲,分布式账本有潜力改善现有金融系统,但是依然缺少成熟度,需要在法律、合规等方面进一步的评估和加强。

这三家央行的行动表明用 PFMI 原则衡量分布式账本在金融系统中应用的重要性。在分布式账本在金融系统的应用问题上,合规性相当重要。从现有的技术成熟度来看,目前还有很多问题,离实际使用还有差距。业界普遍认为扩展性是分布式账本最大的问题,却不知道加拿大央行对透明度要求更高,不能快速找到账户和交易信息,系统就是能扩展也不行。在这些央行实验中遇到的问题不是化妆式(cosmetic)的问题,而是结构性(structural)的问题。结构性的问题不是一两天就能解决,也不是一两个月就能够解决,结构性的问题可能一两年都无法解决,而且解决方案有可能是重新做一个系统,因为原来系统结构不能匹配PFMI原则。
